I'd like to explore why I've found the new Blue Zones documentary so interesting. I think its actually a little bit repulsive thinking of the rich middle-classers looking for small ways to reshape their current lifestyle to live to the hundreds. Or even glorifying the blue zones like escapism from their prison of comforts. For me, living to an old age, and most of those 100 year olds in the documentary have more vitality than I do at 27. That's not an exageration. For me a long healthy life is not the goal but success on a deeper level.

I will say that I've felt more vitality in these last few years since I stopped trying to make someone else's goals my own. It's a common pattern I see and others that I was afflicted with myself. Others like Uncle Ted may have the best intuition for why its so common.

Since I can remember I've had great ambition. After college I channeled that ambition into my first full-time job. But there was from the start a conflict because I chose that job not just because of my ambition but also because of the prospects of making money. A big salary is quite exciting to a college graduate. That and the promise to work on exciting tech that is going to change the world is enough of a promise to jump on board.

But soon I realized the company had no potential. It was a dead idea in my eyes but I kept on for the salary. In the start up world it is heresy to not fully believe and identify with your work. So I had to outwardly express excitement and inwardly tame my doubts. I was working against my self and my own potential by this point.

When this company internally imploded I was extremely relieved. As well as confused and jaded from the experience. But my hopes were not gone and I remember particularly being excited by Polkadot, because of the company structure of small fluid groups. Embodying anarchy at the corporate level. I learned to program in rust to prepare to work there. And picked up my first contracting gig once I knew enough rust. It payed well and I rather liked setting my own hours.

But the dissonance from my old job was still there, between my belief in the company to do anything meaningful and the money they paid. I liked contracting for this reason. I didn't feel like a heretic for not concealing my disinterest in the product as well. But still an interest was needed for me to innovate on it. And I found myself spending a large chunk of my time searching for an interesting angle, trying to identify with the vision of the company so I could play along.

Soon again the company atrophied as did I and went our seperate ways. My next contract was with a more successful start up but my lack of belief in the mission was the same. I worked on two seperate occasions, was offered a full time position but by this point I knew better than to think I could manage the mental dissonance that would entail. So I can say I was learning, albeit very slowly over years.

My friend was working at a startup, their first employee, and had me check out the whitepaper. The insight in that document was profound, both deeply technical yet simple, elegant, and with purpose. Before I knew it that's where I was working full-time, as I thought I never would again. As the soon to be (for a little while) only employee. I stayed there for one year before quitting due to dissonance.

But I wouldn't take that year back if I could. Although I was paid and the CEO younger than me, it was an apprenticeship. Our work was bold, simple code and complex ideas. Assignments felt more like homework in the universities, where we did things ourselves without the use of big fancy frameworks. And yet in that year we accomplished more functional software than I thought possible. This was a wake up to me, and confirmation of my previous dissonance, that the theatrics of startups with their partnerships, using fancy complex software frameworks, is not necessary. In fact its mostly just that, theatrics.

Although I was now working on a meaningful project, I felt a disconnect between what I believed was necessary action and what I was doing. I was building a system. After a year that system was mostly built and we were on to building instruments on top of it, improving infrastructure and the rest. That was one reason I lost connection with the work. In a sense I had identified myself so strongly with the vision that I no longer believed our work was serving it to the best of our ability. I remember being interested in disseminating the technology, introducing it to people that could really make good use out of it now. I felt we were too stuck in the software and not connected enough to real people to put this already functioning system to use.
