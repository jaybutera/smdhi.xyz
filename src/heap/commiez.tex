There is clearly some change that needs to happen in the current societal model of ..? Maybe in an ideal utopia we could completely separate ourselves and live in an isolated dropout sub-society which does not rely on the fucked-up one at all. And lots of people do this with ecovillages, which rely on money still but not that much. They don't need that much to survive other than the vast land these permaculture sites own. Not getting into the philosophy of what it means to be completely independent. Its fair to say many ecovillages just don't have the same need for money that, say a coliving house in the city does.

But such places are so separatist they require you to 

Honestly though I really don't know what I'm talking about when it comes to ecovillages. I've never visited one, and information online is pretty sparse on the gritty details. I don't know how much food they produce and how much they import. And what they need money for or how they're affected by the seasons or yearly changes. I don't know enought about them to talk about it or judge it. For all I know they could be doing exactly what I am wanting to see.

My whole point for even starting this writing session is to jot down ideas of an integrated sub-society which has inflow energy from the mainstream and also an internal culture and community which is at least partially self-sustaining. This is achieved through a synergy between the bringers-in of money - the digital nomads, who work remote for companies with funding from the mainstream - and the community-builders who are locally focused people actually living in a sub-society which is partially funded by the money brought from outside.

Now this description could be of any expat country - Vietnam, Thailand, Bali. Where digital nomads go for cheap housing and sit at service houses all day drinking coffee, eating avocado toast and working for their remote companies or their online business. The difference is that these places are not encouraging local culture, in fact they are destroying it. Gentrification, money from the nomads goes to to the apartment ownersa and the shop owners, who are very few, often businessmen who likely don't even live locally either. The only locals are the workers who are paid cheaply for (what would be way above local price) accomodation and food the nomads pay.

The reality is even digital nomads (who I'm considering different from expats in that they travel almost as an identity, searching for beautiful experiences, connection, and a sense of freedom) are jaded by gentrification. The travel is mostly to escape the capitalist mundane and seek deeper connection. The cacao ceremonies; often superficial and run by yet another digital nomad rather than any local. And no need to contemplate the locals' situation who are often lucky to get a job serving coffee at the nomad service-houses.

Both ends of this nomad-system could benefit from a different arrangement. One could be reached if the two groups work together, in a sense cutting out the middlemen which would be, simplified - the landowners. First of all the locals need to benefit from the nomads' wealth. Right now in the nomads regions - Chiang Mai, Rishikesh, Ubud - a few landlords are benefitting massively but the majority of locals are not. Despite the wealth of the people they are working for.

The locals, who are the community-builders because they are making the local community possible by providing their own labor (be it haircuts, coffee service, construction, coffeebean dilvery), should benefit proportionally to the wealth of the community. That is, all the money coming in to Chiang Mai or wherever. There are likely many ways to achieve this but we have a constraint which is that it should be without central top-down control.

In Chiapas the Zapatistas have used their own currency, the Tumin, for many years. Locals vendors accept it at the markets, often with a discount as a "local price". Such a grassroots currency development would be useful with the nomads in mind. To offer services only within a community-currency so that the bringers-in must purchase it and operate under its terms. Just getting some vendors to use this currency would not be enough because they would need to compete with other vendors who accept the old cash. Again, we aren't trying to form a worker's union so the locals can raise prices. That is much too simple and would never work as there would always be a tragedy of the commons; some vendor not cooperating. And it would not work either because, again, the locals are either selling to other locals (think cheap pad thai with plastic tables), in which case competition makes even the foreigner's price cheap. Or they work at a place which targets the nomad/expats and don't actually own the facility to even accept a different currency.

Shops that choose to operate in a community-currency would have to benefit from it more than accepting cash. They would not make more in profit than they would accepting a currency which everyone already uses. But there would be other benefits to being a member of the community-cooperative.

Radically shifting the dynamic of power in these foreign-funded local communities is going to require a radical shift in consciousness of the communities. One cannot expect to make more money to pay for the same things by changing the accepted currency. There has to be a community mindset. Which means people changing how and where they live. Perhaps where their food comes from (or who provides it). At first it would be small but over time successes have the capacity to shift the collective expectation of lifestyle. A local family which pays a significant chunk of its income for an apartment may find it unthinkable to share closer co-living for instance with other families. But if an identity of community were to be built first, and there were success in outcomes its not unthinkable to imagine a very different future.

So we come back to how such a community would achieve partial independence from the mainstream. By creating a community which is separated - just as a thought experiment, an ecovillage with emphasis on nature - its possible to bring in new value (natural lifestyle) which is not addressed in the mainstream system of profit as it exists in these foreign-funded towns. The creation of new value is worth the change for some. And same goes for the bringer-ins who would benefit from a stronger sense of community. But the nomad communities are global. Unconstrained by geography. Its a world where friends travel across the world to meet other friends on a whim. But community still exists, it is just removed from localities.

Local communities would benefit from a non-local identity as well. The nomad towns are already a global network to the nomads, but not to the locals. Such a network would enable the locals and nomads to cooperate. And both could benefit greatly, the nomads would benefit in being part of a locally-driven community, but on a global scale. The locals benefit from the global cooperation and direct capture of wealth.

Nomads often want more local foods, humanely produced. Products without pesticides. This is possible if there were a large-scale coordination to fund communities to provide such things. Local communities can't form on their own with the land already overpriced from foreign investment and nomad-level prices. But the nomads could afford to start local operations and share in
