---
title: 'Market Failure Among Crypto Capitalists'
description: ''
date: '2021-09-26'
published: true
---

There seems to be a failing within the cryptocurrency community to produce meaningful projects. Looking at the focus of defi projects today, its hard to draw a connection to the original cypherpunk movement which started the cryptocurrency revolution in the first place. Most crypto-enthusiasts would classify themselves as radical libertarians and use the associated platform to justify acting selfishly in the emerging markets. There is an assumed premise that in a free market everyone can act purely for their own benefit and by the means of producers meeting consumers, the things we want built will be produced.

The narrative I hear to sell this premise goes like this: "the current state of the crypto markets is like the dot-com bubble of the early 2000's. So many companies were produced it was hard to tell what was useful and what was pyramid schemes or garbage. After the bubble popped and the markets crashed, the only companies to remain were the truly useful ones, like Amazon and Google." As if we want to see cryptocurrency follow the same fate as the tech monopolies...

I suppose one might argue that crypto is different because it supports a free market and ownership is decentralized so it won't be used as a tool for exploitation like tech companies tend to do. But this argument fails to recognize that cryptocurrency is a technology and technology is developed by us. There is nothing inherently good about cryptocurrency, hence Facebook can make their Libra coin look as evil as Ecoin from Mr. Robot. And we don't have any correct-by-construction proofs that the crypto markets will produce exactly the technology we want to see.

There is such a thing as a market failure. That is, the circumstance where everyone acts completely rationally to maximize their own benefit, yet the outcome is worse for all. The most basic example of a market failure is in economics 101 - the Prisoner's Dilemma. Two suspects in separate interrogation rooms are inclined to rat each other out in order to minimize their own sentence and hedge the risk of being ratted out by the other. If neither said anything, they could both go free. Instead they are both sentenced on a full charge.

There are more practical examples of market failures. They arise when property is difficult to divide privately. For example a community's lake may be polluted, but because everyone benefits from the lake being cleaned, each individual may be inclined to wait for someone else to do it. Even if a group tries to pool funds and gather support, I as an individual may do better to not get involved and let them clean the lake anyways. And the group fails to gather funds.

Just like with Linux there are many projects which are very important and many people agree would be useful, but they are not easily monetizable. Vitalik Buterin talks about this a lot in considering how to fund public goods like documentation, research, and Layer 1 development.

It's important to clarify, I'm not saying free-market capitalism has failed in cryptocurrency. On the contrary, miners/stakers acting purely in their own self interest seems to work very well for the security of blockchain consensus when the protocol is designed well. But free markets working for this highly specified case doesn't mean they will work just as well everywhere, and that it should be adopted as a sort of universal "moral philosophy" where individuals can always act purely out of self-interest just because the system they are working in is free of coercion.

Yes we now have a currency which requires much less trust than the previous model. And yes we now have a world computer whose programs we can rely on to execute faithfully. Now the attention seems to be shifting away from meaningful tools for society, towards optimizing investments on these platforms, minimizing transaction fees, developing competitive arbitrage. Will this movement devolve into a flurry of fintech companies? There are so many possible revolutions to be realized if only we care enough to take part.
