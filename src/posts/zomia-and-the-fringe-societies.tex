---
title: 'Zomia and the Fringe Societies'
description: 'Why is it interesting to learn about the different Zomias and fringe societies of resistance from the states? I think the revolutionaries of the world can benefit from this identity. To identify with the sovereign, with an escape from the system rather than a modification of it. To refuse the world systems at large and accept their independence, not from someone else like the more powerful nation-states. I want to see people take their lives and their communitys'' lives into their own hands. Have a sense of independence which cannot be given by foreign aid outside.'
date: '2023-08-16'
published: true
---

Why is it interesting to learn about the different Zomias and fringe societies of resistance from the states? I think the revolutionaries of the world can benefit from this identity. To identify with the sovereign, with an escape from the system rather than a modification of it. To refuse the world systems at large and accept their independence, not from someone else like the more powerful nation-states. I want to see people take their lives and their communitys' lives into their own hands. Have a sense of independence which cannot be given by foreign aid outside.

And this is not nationalism. Because that would be a takeover of the state. I want people to identify with the exit from the whole system. Refuse to participate in the perpetuation of the state oppression. Take responsibility for themselves and their community. And to love and recognize other peoples' capacity to do the same.

Because a nationalist uprising will only bring more of the same samsara as it has. Switching powers from US and Europe to Russia and China taking control of Africa. And so Africa demands independence but what does independence look like? Is it a federation of totalitarian states? Mussolinis in order to crush the outside influence of France and even of Russia?

We need strength to demand our independence but not oppression. We must rid ourselves of the state mindset. The societies removed from civilization by natural mountain division - the hill people - have escaped their land but only because they didn't have the strength (many failed revolts). But they show strength in the hills through their persistence and commitment to their lifestyle on the fringe. For their young to grow up learning their own language and not leaving to study in the cities (although that may be less the case now than it was in the past).

The strength I want to see is not from oppression but from commitment to community and by working with the strength of nature. The mountains, the marshes and the forests are nature's fortress. And now we have the internet where communities are taking refuge under encryption, cryptocurrency. Technology too can be a non-oppressive strength.

All this talk of non-oppression it would be valuable to define it. Oppression in the context of the state-mindset is to view life as nothing more than a material object and to treat it as such. The state is thus a slave institution. To acknowledge the life of a tree or a human is empathy or compassion. This is the realm of consciousness which transcends the foundational model of statehood and oppression.

Sometimes defense is natural and can be done from a love for life. Not love of nation or ethnic identity but because life is being encroached upon and life naturally defends itself like a rose thorn. The Bhagavad Gita afterall, is a lesson on how to fight a war with total and unconditional compassion. Complete internal peace despite the external chaos. And the result is the cessation of new karmic cycles. Arjun shows up on the battle day and decides he doesn't want to fight his own brothers but Krishna says its too late for that. And shows him how to fight with internal peace.

I actually believe these hill people who are refugees of oppression are a wonderful embodiment of the natural responses. They were neither destroyed like a beetle under a shoe, nor assimilated into the state system of oppression. To chase money and power and generally move away from compassion and recognition of life - to a material-only mode of consciousness. This is why they are worthy of our study and modelling. To learn from them and copy them where we find their traits to be good and valuable. And through studying them to realize that the state does not have the global dominium people assume it does. Not only is it possible to opt out but there has and is a constant dialectic between state and non-state society.
