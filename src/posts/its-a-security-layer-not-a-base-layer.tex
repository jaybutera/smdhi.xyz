---
title: 'Its A Security Layer Not A Base Layer'
description: ''
date: '2022-02-21'
published: true
---

If you haven't heard of the [Fat Protocol thesis](https://medium.com/ledgercapital/the-fat-protocol-thesis-debated-65ad56285fd5) before, it goes something like this: traditional technolological protocols like HTTP and IP tend to lose value to the application layers like Google and Amazon, even though these applications all run on the base. The blockchain method of tokenizing protocols allows value accruation to move back to the base layer. The premise for this is that each application on top of the base protocol will bring wealth in terms of transaction fees and storage to the base layer, and that the economic activity on the base will grow proportionally with any application built on top. The thesis also proposes that shared data on the base layer contributes a network effect to each application.

This is a particularly important thesis right now since the negative effects of a fat protocol are essentially driving the "layer 1 alternative" revolution. Many projects are now competing for market share as a base for applications. In considering Ethereum's history we can see the fat protocol form. First ERC-20 "virtual" tokens were established within smart contracts. Then exchanges like UniSwap provided liquidity to the tokens. Next yield farming applications provided liquidity to the exchanges. Each application operates with the Eth token. Eth becomes more valuable as its demand grows in proportion to all of the applications on top. Eth is also an object of speculation, value and attention accrue at the base Ethereum protocol level as a hedge on the ecosystem as a whole. The demand for the Ethereum token is directly correlated with the use of applications built on it.

Interestingly the original fat protocol proposal is a positive narrative, proposing that cryptocurrency may be a solution to funding core interoperable protocols. But taken to its natural conclusion, blockchain protocols could be seen as a pyramid scheme where applications are locked in to paying portions of their wealth to a common base, and investment and speculation are extracted from new projects back into the base protocol. Furthermore transaction fees become prohibitively expensive and applications cancel eachother out. What started as a network effect turns into a deadening of the whole ecosystem.

Crypto billionare [Su Zhu on the Uncommon Core podcast](https://uncommoncore.co/podcast/) paints alternative layer one blockchains essentially as a populace revolution against the crypto-elites of Ethereum. Unfortunately these alt-chains suffer the same failure to scale the ecosystem under the same model. The alternatives are not addressing the "fat protocol problem", but instead they iteratively improve older tech, and mostly operate better only because of smaller user bases and/or more centralization. It's like we have found an O(n) scaling solution and the new layer 1 blockchains are offering O(K*n). It's the same algorithm and the same scaling properties.

The fat protocol model is anti-scalable. Unrelated applications inherently affect one another. Each new application makes every other application more expensive to use, since each application must pay a proportionate amount of its use to the base in transaction fees. Communities end up needing to fork once they grow large enough in order to scale. As blockchains fragment into application-specific domains, we fail to meet Eric Tung's core thesis of [trust as the core value proposition of the blockchain](http://jaybutera.info/posts/trust-is-the-core-value-proposition-of-a-blockchain.html). It is important that our security-critical applications rely on a trust-worthy foundation, or else we fall back into needing to trust centralized organizations to manage information again. An ecosystem whose trust is divided into application specific domains where each domain bootstraps its own security is as easy to conquer as the size of the domain itself.


It would be a shortcoming not to acknowledge the fundraising aspect of cryptocurrency projects. In fact I think it is a common viewpoint that "trust" is only the thing that gets us to the thing. And that thing is a way to organize people around a project and express value. Allowing people to share stake in a collective project. This assessment may seem unfair. Afterall, some of the most important applications for blockchains will be simple public-key infrastructure which is tangential to any fancy "fundraising" techniques. But it is undeniable that crypto-tech today would not be where it is if not for the markets inherently embedded within it. Raising funds seems to be a core energizing aspect to the crypto movement.

If the fat protocol breaks at scale, do we lose the ability to raise funds for all crypto projects? I don't think so, there are DAOs where applications do not require the value-signaling token to operate, but the token still captures value of the DAO. MakerDAO for instance, uses its token MKR for stability by arbitrage, but applications using the stable coin DAI are not dependent on the price of the MKR token.

So the question becomes, can we build secure platforms which scale with growing application ecosystems, without stealing speculative value from them, and still fundraise development of the platform itself? To achieve this it seems that applications need to decouple their magnitude of operation from the use of a base protocol, meaning use of the base stays constant with the growth of the application. Rollups are a direct solution to this need. In fact optimistic rollups may solve the decoupling, with a compromise to security. ZK-rollups compress transactions but still store everything on the base layer, meaning fees proportional to the use of the rollup must be paid to the base.

Platforms can also improve this decoupling. The Themelio layer 1 blockchain for instance uses a stablecoin, MEL, for transaction fees. Thus growing demand for MEL will not increase its value, minimizing speculative value accruation at the base which may otherwise go to applications.

To recap, layer 1 blockchains which "host" the entire application ecosytstem on them are inherently exploitative of the applications at scale. In fact, we should stop calling layer 1 blockchains a "base layer" and instead call it the security layer. Because the intention is not to build a pyramid of blockchains; but to have a trustworthy anchor for the trust-critical parts of an application. It seems the best way forward is to focus on decoupling application usage from the security layer. The alternative is to keep building technology that scales poorly and further fragments security in the ecosystem, in the name of short-term capital raising.

It is an open question what kind of applications and tokenomic project-funding models can be created and how to utilize a security layer effectively and efficiently. Failure to rise to this challenge may decide the fate of this crypto-revolution to be hijacked by old systems of power which are constantly testing the weak points of this technology. Particularly with the recent developments in cryptography such as zero-knowledge proof systems, I am optimistic that solutions exist to decouple applications from their usage on a security layer. I am more concerned that the narratives surrounding crypto are focused on small incremental improvements with immediate gains than they are of solving big picture concerns of security and a sustainable user ecosystem.
