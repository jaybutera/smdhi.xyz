---
title: 'Ends In Capitalism'
description: ''
date: '2021-08-21'
published: true
---

In David Friedman's book "Machinery of Freedom", chapter 33, he claims that "the organization of a capitalist society implicitly assumes that different people have different ends." While it's certainly true that people have different desires and wishes, I don't think its necessarily true that people have different ends. And it is not a clear assumption of capitalism either.

To distinguish between an end and a desire; an end is something you want for no other reason than the end itself, and a desire in this context is something you want in order to achieve an end directly or indirectly. David Friedman says that socialism assumes there is an agreement on the goals of the society. Everyone works toward the "common good". He therefore puts capitalism in opposition to this by saying that capitalism assumes different ends for each person. While I agree that capitalism is in contrast to the socialist assumption of a common end, I think how he claims capitalism differs is subtly wrong.

People can have a common end but disagree on how to get there. This is what capitalism supports; the plethora of ideas on how to reach what I believe is a shared end among all people. You can call that world peace or happiness, I'll call it simply getting some rest.

So why support people's conflicting beliefs on how to reach the same destination? Why not try to bring people together to agree upon and share a vision for how to reach an end they all agree on? Capitalism does not oppose cooperation and shared visions. Such organizations can exist just fine in a capitalist society. Different religions, non-profits and volunteer groups are all examples of people cooperating and sharing values to reach a common end in a not-necessarily capitalistic way.

But the same thing about a socialist society cannot be said. If a set of people agree on some means to achieve an end, they don't need an institution to tell them what to do because they'll do it willingly. Only in the case that someone in that group exploits the work of the others for themself is an institution necessary. Because it is assumed there will be someone that acts selfishly in this way, some form of coersion is needed on everyone and thus a socialist society is born.

The other axiom I will suppose besides that everyone wants the same end is that most people's ideas of how to get to that end - happiness - are wrong. As a dark example Hitler thought genocide would lead to happiness for the people that remained. Someone must decide the means toward a common goal in socialism. Even democratic socialsm signals the average belief which is most often wrong. It is better to support the freedom of individual people with differing, potentially conflicting, beliefs on how to reach the same end than it is to coerce everyone into a specific path that is chosen by someone for someone else.

Of course supporting the individual freedoms of people to make these mistakes means that holy wars will happen. But this is natural. Unlike the compulsive mentality of a coercive socialist society which tries to, so to speak, suppress the weeds from sprouting through the concrete. It is not the job of capitalism to create a beautiful shared world. That is left to us in whatever ways we see fit. Capitalism can be left to simply do what it does best, which is to generate resources (through a free market) for people to execute their means. It does this even while assuming selfish actors, who simply want to maximize profits for themselves but in doing so end up providing whatever it is people believe they need. What those beliefs are is not the concern of capitalism.
