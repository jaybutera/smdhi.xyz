---
title: 'Trust is the Core Value Proposition of a Blockchain'
description: ''
date: '2021-09-30'
published: true
---

We have found a lot of applications for blockchains in the last decade. They aren't just currencies anymore, they are shared computers which host the future technology of financial markets, serve as digital identity stores, and provide the foundation for what we now call Web3. With so many applications, the true value proposition of the blockchain may be muddled. One might say the value is in decentralization. Any applications built in a blockchain inherit the decentralized property of the blockchain itself.

If decentralization were truly the end objective of our systems, then it would be much cheaper to simply drop digital identities in an IPFS file and disseminate that on the sharing network. But this doesn't work because it requires trust in whoever creates the file to provide the right information. Hence we get to the root value proposition of a blockchain, a source of information which we don't need to trust because we can be confident in its accuracy. It's easier to be confident that a blockchain is working when it is decentralized (I don't trust an enterprise Hyperledger instance more than I trust who is running it) but the decentralization is only a strategy to achieve the former end.

We can call this end, the trustless-ness aspect of a blockchain, as a form of minimized trust. We aren't trusting the validators of a protocol. We only expect them to act in their own self-interest, and we trust the cryptoeconomic reasoning for why the functioning of the protocol aligns with their own self-interests. The blockchain becomes a root of trust.

Not all blockchains are equal. While all aim to preserve some level of trustless security, the design decisions often simultaneously optimize for performance and computational expressiveness, sometimes to the detriment of security of the "root of trust". On the other hand, some blockchains lack expressivity and so miss many critical applications for a programmable root of trust.

Consider Bitcoin, it was designed to be an electronic cash system but largely fails to do that because of its expensive gas fees and long confirmation times. The Bitcoin community has recognized the paradox of protocol upgrades in a system that needs to minimize trust. Essentially upgrades are not part of the cryptoeconomic reasoning of Bitcoin and so compromise the integrity of that core root of trust with unknown variables. Who decides a protocol upgrade? What kind of governance process can be trusted to be "outside" this root of trust?

On the other hand the Bitcoin protocol lacks crucial features which actually force users to introduce trust at the application level. These sort of leaky abstractions cannot be patched over. For instance, there is no concept of finalizing a block in Nakamoto consensus. As a result it is possiible in an attack or a network failure that blocks may be reverted. Consider a trustless bridge application that swaps Bitcoin for Eth. An arbitrary "good enough" number of blocks to consider a Bitcoin transaction safe must be hardcoded into the decentralized protocol. But having a significant number of blocks be reverted on the Bitcoin blockchain is not unheard of. Likely, many applications don't even account for the possibility of this failure case. As Satoshi points out in the Bitcoin whitepaper, "With the possibility of reversal, the need for trust spreads."

Most other blockchains have decided to embrace upgrades to some degree. People speculate that usability or performance or "the killer application" are what is needed for cryptocurrencies to achieve mass adoption. But the widest-spread cryptocurrency of all is simple and sluggish. As Eric Dong, inventor of the Themelio blockchain, points out [trust is the killer application](https://docs.themelio.org/basic-concepts/01-endogenous-trust/). The trust introduced by regular upgrades is the actual barrier to adopting the average blockchain. If Coca-Cola wants to handle invoices on-chain, or Detroit wants to use NFTs as an authoritative deed for home ownership, they need to decide which blockchain to use. To choose a more lean "startup" chain like Stellar or Cardano is essentially to make a deal with a 3rd party business. It requires trust in the development of that chain over time and the people behind that development.

Even Ethereum doesn't seem ready to act as the backbone of America's housing ownership database. Even if the alternative is a slow and inefficient government body. Although the development process is less centralized than most, Ethereum still suffers from unpredictable changes and leaky abstractions. For instance many defi applications that currently exist on-chain will be affected by the introduction to data sharding. In the near future, defi applications may benefit from accounting for which shard they are on and only communicating with other applications in the same shard. Completely trustless apps which don't have root keys won't be able to upgrade and so may become obsolete. This may be a contributing factor to why "DAO" projects, projects which introduce governance to the protocol, are so popular on Ethereum. To account for the unpredictable changes from underneath.

Another property of cash that Bitcoin unfortunately fails to replicate is a stable value. Someone that wants to send money across a border or in some uncontrolled, unmediated fashion, would only resort to Bitcoin if they were so desperate that they were willing to risk losing 5% of value in the time it takes to convert a stable currency into Bitcoin and back again on the other side. Trustless, uncontrolled cash is a huge value proposition that is unfortunately lost in cryptocurrency today. The best we have are stable coins that peg or track fiat.

Stable value is not only useful for cash, it is also a desirable property for applications and in cryptoeconomic incentive design. Consider Ethereum's Name System, ENS, which offers an alternative root of trust to the traditional Certificate Authority infrastructure of the web. Again the value proposition is massive, traditional DNS has many security vulnerabilities which make it easy for governments, ISPs, and authoritative companies to break. But the buying and selling of ENS names became entangled with the fluctuation in the value of Eth itself. An ENS name worth $5 today may be worth $50 in half a year. So last year  to price names in USD, using oracles by Chainlink. That's quite the mouthful of external trust in that last sentence. But the unfortunate reality is that there was no better option at the time to fix the economic incentives of ENS. 

Infact Mozilla's web standards lead [rejected the W3C Decentralized Identities proposal](https://lists.w3.org/Archives/Public/public-new-work/2021Sep/0000.html) just days ago, partially on the grounds that "The DID architectural approach appears to encourage divergence rather than convergence & interoperability." He is alluding to the spec registry which contains sources from a variety of blockchains for DID, 15 of which belong to different applications on Ethereum alone ([one is ENS](https://www.w3.org/TR/did-spec-registries/#did-methods)). This issue could likely be alleviated by optimizing not for diversity but instead by measuring the entries in terms of the most necessary property; minimized trust. The resultant registry would be much smaller.

It is not unreasonable to expect future versions of our most critical security protocols to use a blockchain as the root of trust. TLS, the protocol which encrypts our internet traffic, may one day resolve trusted certificates from a blockchain instead of a certificate authority. But only if we can make a blockchain which we are confident will be more secure, more impartial, and more reliable than the existing certificate authorities of today.

Beyond internet protocols, there are more tangible applications for a sound root of trust. Courts can use a blockchain as an authoritative source to host the rulings of a case. Smart contracts, [as first described by Nick Szabo in 1997](https://fon.hum.uva.nl/rob/Courses/InformationInSpeech/CDROM/Literature/LOTwinterschool2006/szabo.best.vwh.net/idea.html), can use this information on chain to make further decisions and enforce rulings of a court. Note that courts here don't have to be part of a national government. Companies use 3rd party arbitrators all the time as a more efficient way of managing contractual disputes between companies internally. It is also necessary to decide on a trusted arbitrator in the case of cross-border contracts, where the transaction exists outside the jurisdiction of any particular nation. These may seem like edge cases in small markets but the reality is that a secure and cheap alternative to dispute resolution could change every market that exists today from futures contracts with farmers and producers, to defi and on-chain insurance.

Many thanks to Eric Dong, most of these ideas about blockchains as a root of trust and problems with governance can be found in his articles, [Against Blockchain Governance](https://nullchinchilla.me/posts/against-blockchain-governance.html).
